
% Part 1: history and background

\epigraph{The beauty of a living thing is not the atoms that go into it, but the way those atoms are put together.}{Carl Sagan \cite{Sagan1980}}

While carbon-based crystals like graphite or diamonds have been of interest to humanity since at least the $4^\mathrm{th}$ millennial BC, graphene, a single layer of carbon atoms ordered in a two-dimensional honeycomb lattice, has a much younger history \cite{Singh2011}. The first scientist who unknowingly came into contact with this fascinating material was the British chemist Benjamin Brodie (1817-1880). He created and analysed graphite oxide, an aggregate of several layers of graphene with hydrogen and oxygen compounds, which he called \textit{carbonic acid} \cite{Geim2012GraphenePrehistory}. It was not until 1962, when the chemist Hanns-Peter Boehm (born 1928) and the physicist Ulrich Hofmann (born 1931) were the first to observe single layers of graphite oxide. Approximately twenty years later, they were also the first scientists to come up with the term \textit{graphene}, a definition that is used to this day \cite{Geim2012GraphenePrehistory,Dreyer2010}. Boehm's and Hofmann's work on graphite oxide paved the way for a lot of research conducted in the following decades, hoping to find admirable electrical properties \cite{Singh2011}. However, scientists of all fields struggled with the reduction of graphite oxide to mono-layer graphene, let alone the creation of a single layer \cite{Singh2011}. On top of that, theoretical physicists argued that graphene could not exist due to the fact that two-dimensional crystals were thermodynamically unstable \cite{Singh2011, Geim2007}. Nevertheless, in 2004 a paper \cite{Novoselov2004} published by the Russian-born physicists Andre Geim (born 1958) and Konstantin Novoselov (born 1974), showed that the experimental creation of graphene was possible on top of non-crystalline substrates \cite{Geim2007}. With a rather straightforward approach, where they repeatedly peeled off layers of industrial graphite until they obtained a single layer of graphene, they were also able to analyse its electrical properties for the first time \cite{Singh2011, Novoselov2004}. Up to this point, most of the research conducted on graphene-like samples was purely observational, but Geim and Novoselov were able to show that graphene exhibits some surprising electronic properties \cite{Geim2012GraphenePrehistory, Novoselov2004}. The most astonishing discovery was the fact that electrons could travel thousands of interatomic distances without being scattered, even though the graphene layer was analysed in suboptimal conditions (rough substrate, room temperature, polymer residue), which implied that the electrons in this two-dimensional material were basically unaffected by the non-homogeneous environment \cite{Geim2012GraphenePrehistory,Geim2007, Novoselov2004}. These completely counter-intuitive electronic properties sparked a new interest in this material and the graphene \textit{'gold rush'} was initiated \cite{Geim2007}. Throughout the last two decades the number of publications mentioning graphene has grown exponentially and not only have new electronic properties been discovered, but also astonishing thermal, optical and mechanical features came to light \cite{Singh2011}. In 2010, Novoselov and Geim won the Nobel prize 'for ground breaking experiments regarding the two-dimensional material graphene' \cite{Singh2011}. The scientific community hopes for many future applications of graphene like field effect devices (e.g. transistors in computers and smartphones), sensors, solar/smart cells, energy storage devices, composites -- to name a few \cite{Singh2011}.\medskip 

% Part 2: why are we investigating this?

% Plasmons

One way to analyse these remarkable electronic properties of mono-layer graphene are plasmonic excitations (or plasmons), which are collective oscillations of the charge density \cite{Bao2017}. These plasmons have some appealing properties, like their strong interaction with electromagnetic radiation \cite{Bao2017}. Especially in graphene, which has a high carrier mobility and variable charge density, plasmonic excitations with unusual properties can be found \cite{Bao2017}. Plasmons have already been investigated for a long time in metals, but due to the strong energy loss of free moving electrons in metals, it is very difficult to control these charge density oscillations \cite{Bao2017}. This is different for graphene, where plasmons can easily be tuned with different environments (e.g. substrates) \cite{Bao2017,Huang2016}. Furthermore, unlike plasmons in metals, graphene plasmons exhibit a strong energy confinement \cite{Huang2016}. These are strong indicators why graphene plasmonics could find a large number of applications in fields like photonics, optoelectronics, molecular/biological sensing or light detection \cite{Huang2016}.\medskip

% Coulomb

Based on the definition of plasmonic excitations, it is not surprising that they highly depend on the Coulomb interaction between electrons. In solid state physics it is sometimes common practice to only regard atom-atom and electron-atom interaction, but this can lead to quantitatively unjustifiable systems \cite{Czycholl2016}. The electron-electron interaction, especially in two-dimensional materials like graphene, can be very strong and should not be neglected \cite{Noori2019}. This so-called screened Coulomb interaction has been studied to a large extent. However, research examining the effects of the environment (dielectric screening) on the screened Coulomb interaction is still in its early developments \cite{Noori2019}. \medskip

% Why?

It is our goal to investigate the influence of dielectric screening on the plasmonic excitations of graphene. A two-dimensional substrate which has been of interest in the last years is hexagonal boron nitride (h-BN), which due to its structural similarity to graphene, is expected to generate less disorder compared to other substrates (like silicon dioxide) \cite{Sarma2011}. The effect of environmental screening has been done to some extent in momentum-space, which can have computational advantages \cite{Roesner2015,Hwang2007}. However, as we are interested in analysing plasmonic excitations and displaying them on a graphene lattice in Cartesian coordinates, it can be beneficial to work in real-space \cite{Westerhout2018,Westerhout2021}. Research addressing the dielectric screening of the environment in two-dimensional materials and analysing the resulting plasmonic excitations in real space has only been done to a small extent \cite{Westerhout2021,Wang2015} and will be the main goal of this thesis. \medskip

% Part 3: description of thesis

We will start by introducing the theoretical background (see section \ref{theory_plasmons_coul}) of two important quantities for the derivation of the dielectric function (see section \ref{plasmonic_excitations}): the tight binding Hamiltonian (section \ref{tight_binding_method}) and the screened Coulomb matrix (see \ref{screened_coul_inter}). These concepts will be applied to some extent to a simple square lattice (see \ref{square_lattice}). We will then use high-quality ab initio calculations based on the constrained random-phase approximation (cRPA) to create a model which describes the screened coulomb interaction in mono-layer graphene (see section \ref{monolayer_graphene}). This model can in turn be applied to derive the dielectric function in graphene, which will be the basis of describing real-space plasmonic excitations. Using the model found for free-standing graphene, we can analyse the effect of dielectric screening from the environment by comparing various approaches with the ab initio data for h-BN (see section \ref{boron_nitride}). Once again it is our goal to find a suitable model which represents the Coulomb interaction within mono-layer graphene and to describe plasmonic excitations within this surface. Last but not least, the effect of different dielectric environments will be examined to investigate the effect of environmental screening on the plasmon frequencies.

