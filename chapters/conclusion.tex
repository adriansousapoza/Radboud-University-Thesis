
From first principle cRPA calculations, we were able to derive a continuous model for the screened Coulomb interaction in real-space for free-standing graphene. The same model can be applied to graphene embedded in h-BN by altering parameters to account for dielectric screening such that it properly corresponds to the ab initio data. Combining this Coulomb interaction model with a simple tight-binding Hamiltonian, which only accounts for nearest neighbour hopping, we determined the dielectric function. The eigenvalues of the dielectric function could be used to find plasmon eigenmodes (eigenvectors of the dielectric function), which were mapped in real space. Furthermore, we were able to determine how the plasmon frequencies alter when the dielectric environment is changed. Even though the results of the research conducted in this thesis is satisfactory there are a couple of issues that should be addressed. \medskip

One of the most important problems that should be discussed, is the Hamiltonian in equation \ref{eq:hamil} and equation \ref{eq:hamil_coul}, more concretely the hopping parameter $T_{i,j}$. In this thesis the general (and highly inaccurate) assumption has been made that only nearest neighbour hopping has to be considered. In general electron hopping can occur between any two lattice points, but with different probabilities, i.e. different values for $T_{i,j}$ \cite{Westerhout2021}. These values can change the tight binding Hamiltonian quite drastically, which implies that a different dielectric function will be obtained after the RPA calculations, which in turn will lead to different electron energy loss spectra, eigenmode plots and plasmon frequencies. The frequencies in particular cannot be trusted without analysing the dielectric function based on a more accurate tight binding Hamiltonian with more hopping parameters. Nevertheless, the Coulomb interaction can still be analysed as it does not depend on the hopping parameters. Furthermore, the change in frequency for different dielectric constants as seen in section \ref{different_dielectric} will most likely still be visible for a different tight-binding Hamiltonian.\medskip

Furthermore, one has to keep in mind that due to computational limitations, the lattice size of all systems was relatively small. If the ab initio data assumes a periodic lattice, we have to be careful not to accidentally include those effects into the fit for the screened Coulomb interaction (see figures \ref{fig:ab_initio_coul}, \ref{fig:fsg_cho} and equation \ref{eq:cho}). This issue can be resolved by exclusively taking near neighbour interaction into consideration, but that implies less ab initio data points are included in the fitting process. It is therefore advantageous to use larger lattices to be certain that the effect of periodicity can be excluded.\medskip

One of the reason why h-BN is used as an embedding/substrate is because of its similarities to graphene. Analysing different substrates which exhibit dissimilar lattice and electrical properties might be advisable when testing the model for the screened Coulomb interaction (equation \ref{eq:cho}) before using it in different dielectric environments. Furthermore, it might be interesting to see how the interaction model compares to using a substrate instead of an embedding, such that $\epsilon_2\neq\epsilon_3$. The screened Coulomb model in equation \ref{eq:cho} can be used for both cases.\medskip

One parameter that was not discussed in this thesis is the temperature. The conductivity of graphene depends on the surrounding temperature \cite{Sarma2011}. For all calculations in this thesis a temperature of $kT =\SI{0.0256}{eV}$ was assumed, which corresponds to approximately $T=\SI{297}{\kelvin}\approx \SI{23}{\celsius}$, i.e. room temperature. Investigating the dependence on the temperature might lead to new insights, though it is also known that graphene does not have a very strong temperature dependence for $T<\SI{300}{\kelvin}$ \cite{Sarma2011}.\medskip

Nevertheless, the general insights derived from the results of the conducted research can be taken to be positive. One particularly remarkable result were the fitted parameters for the Coulomb interaction in h-BN embedded graphene (see section \ref{boron_nitride}). The fact that the fit interpreted the ab initio data not as three different layers, but as one layer consisting of two materials, came as a surprise. This unanticipated result should be investigated in the future.\medskip

The plasmon frequencies and their corresponding patterns in real space show interesting symmetries ($1s$-, $2s$-, $1p$-, $1d$-like states), which could find applications in many different fields \cite{Bao2017}. Knowing how the dielectric environment influences the plasmon frequencies (see section \ref{different_dielectric}) can be very valuable for future applications.\medskip

To sum up, we were able to derive a suitable Coulomb interaction model in real space for graphene, which takes environmental screening into account. This model can be used for further research which makes use of dielectric materials (e.g. substrates) to examine electrical properties in two-dimensional materials with semiconducting properties. We can furthermore conclude that boron nitride is a very suitable substrate for graphene, as it maintains plasmonic excitations within it. Furthermore we have seen that these plasmonic excitations can still occur in different dielectric environments, where the effect of screening is even larger. These key results are the reason why we can conclude that the screened Coulomb interaction model and the influence of the dielectric environment on plasmonic excitations led to new insights and further investigations can build on the research of this thesis.