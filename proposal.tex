\documentclass{article}

\usepackage{adrian}

%%%%%%%%%%%%%%%%%%%%%%%%%%%%%%%%%%%%

\title{Plasmonic Excitations in Mono and Bi-Layer Graphene}
\author{\vspace{-1mm}Bachelor Student: Adrian A. Sousa-Poza\\
Supervisors: Dr. Malte Rösner and PhD student Tom Westerhout}
\affil{\vspace{-1mm}\textit{Radboud University Nijmegen, Faculty of Science}}
\date{\vspace{-3mm}\today}

\begin{document}

\maketitle

%%%%%%%%%%%%%%%%%%%%%%%%%%%%%%%%%%%%%

\begin{quote}
    \textit{The main goal of this bachelor internship is the examination of plasmonic excitations in mono and bi-layer graphene, specifically taking the Coulomb screening from the environment into account.}
\end{quote}

\section*{Relevant Topics and Procedure}

The following steps are planned to be performed in the given order:

\begin{itemize}
    \item Before jumping right to the twisted bi-layer graphene case it is important to understand more manageable structures, like the simple square lattice, mono-layer graphene and bi-layer graphene.
    \item Of utmost importance is understanding the structure of graphene in both real- and momentum space. The transformation from one to the other space can be done via the (inverse) Fourier transformation.
    \item The electronic band structure of graphene and the square lattice can be approximated within tight-binding theory. This theory assumes that electrons are tightly bound to their corresponding atom, which limits the interaction with surrounding atoms. Based on this model it is necessary to introduce a hopping parameter describing the probability of an electron 'hopping' to a neighbouring site.
    \item To examine the single-particle electronic properties of these models we need to calculate the electronic density of states (DOS), which describes the probability of states being occupied by the system at a certain energy. A high density of states correlates to many states being available for occupation.
    \item Plasmons are collective oscillations of electrons, i.e. intrinsic charge oscillations coupled via Coulomb interaction. Plasmons in graphene, which can be probed with the Electron energy loss spectrum (EELS), records the energy loss of transmitted or reflected electrons. The dispersion relation can then be used to determine the plasmon frequency/eigenmode.
    \item The Coulomb interaction will be be fitted to cRPA derived values from first principles [PRB 92, 085102 (2015)] using a continuous real-space model from PRB 97,041409 (2018). This will be done for both the mono- and the bi-layer case. With the derived parameters and the new model we are able to include the dielectric substrate screening in order to examine the plasmonic excitations more accurately (DOS, EELS, Dispersion Relation).
\end{itemize}

\section*{Schedule and Examination}

The internship started in January and will end in August. The thesis will be handed in before the $15^\mathrm{th}$ of August, such that the supervisor dr. Malte Rösner and second examiner dr. Steffen Wiedmann have enough time to read and grade it before the end of the academic year. The presentation will not be held at the summer symposium, but will instead be given as an online seminar within the department of the theory of condensed matter. It will be held on the $23^\mathrm{rd}$ of August at 13:00.



\end{document}





